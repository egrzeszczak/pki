\documentclass[12pt]{article}
%%  ==================================================
%%  Settings
%%  ==================================================

\usepackage[a4paper, inner=1cm, outer=1cm, top=2cm, bottom=2cm]{geometry}
\usepackage{fontspec}
\usepackage{blindtext}
\usepackage{enumitem}
\usepackage{fancyhdr}
\usepackage[hidelinks]{hyperref}
\usepackage{index}
\usepackage{listings}
\usepackage{color}
\usepackage{graphicx}
\usepackage{caption}
\usepackage{tikz}           % for drawing external link icon

%%  ==================================================
%%  Styles
%%  ==================================================

\setmainfont{Arimo}

\pagestyle{fancy}
\fancyfoot{}
\raggedbottom

\date{}

\renewcommand{\headrulewidth}{1pt}
\renewcommand{\footrulewidth}{0pt}

\definecolor{dkgreen}{rgb}{0,0.6,0}
\definecolor{red}{rgb}{0.9,0.1,0.1}
\definecolor{gray}{rgb}{0.5,0.5,0.5}
\definecolor{mauve}{rgb}{0.58,0,0.82}


\newcommand{\ExternalLink}{%
    \tikz[x=1.2ex, y=1.2ex, baseline=-0.05ex]{% 
        \begin{scope}[x=1ex, y=1ex]
            \clip (-0.1,-0.1) 
                --++ (-0, 1.2) 
                --++ (0.6, 0) 
                --++ (0, -0.6) 
                --++ (0.6, 0) 
                --++ (0, -1);
            \path[draw, 
                line width = 0.5, 
                rounded corners=0.5] 
                (0,0) rectangle (1,1);
        \end{scope}
        \path[draw, line width = 0.5] (0.5, 0.5) 
            -- (1, 1);
        \path[draw, line width = 0.5] (0.6, 1) 
            -- (1, 1) -- (1, 0.6);
        }
    }

\newfontfamily{\codefont}{Input Mono} % You need to have this ttf in your os

\lstset{frame=tb,
  aboveskip=3mm,
  belowskip=3mm,
  showstringspaces=false,
  columns=flexible,
  basicstyle={\codefont\small},
  numbers=none,
  numberstyle=\tiny\color{red},
  keywordstyle=\color{blue},
  commentstyle=\color{dkgreen},
  stringstyle=\color{mauve},
  breaklines=true,
  breakatwhitespace=true,
  tabsize=3
}

\newcommand{\documenttitle}{Symmetric Encryption, Block Ciphers and Stream Ciphers}
\fancyhead[L]{\documenttitle}
\title{\documenttitle}

\begin{document}
\maketitle

Learn the fundamentals of symmetric encryption, as well as the differences between block ciphers and stream ciphers, and their respective modes of operation. This lesson will not go too much in depth to keep these lessons simple, so I highly recommend you to explore each concept in your own time. I'll also provide some resources that I deemed are insightful.

\section{Learn}

\subsection{Symmetric Encryption}

The method where encrypting plaintext into ciphertext and decrypting ciphertext back into plaintext \textbf{using a shared secret} is called symmetric encryption. We can think of a simple scenario where Alice and Bob agree upon a key and an algorithm, and then they use the key to encrypt and decrypt the information they want to share with each other.

% You need `context' package for \externalfigure to work
% \externalfigure[https://upload.wikimedia.org/wikipedia/commons/thumb/8/80/Simple_symmetric_encryption-en.svg/1920px-Simple_symmetric_encryption-en.svg.png] 
\begin{center}
    \includegraphics[width=0.5\textwidth]{resources/symmetric.png}
    \captionof{figure}[LOF entry]{Symmetric encryption process}
    \label{fig:symmetric}
\end{center}

\textit{\blindtext[1]}
\subsection{Block ciphers}
\textit{\blindtext[1]}
\subsection{Stream ciphers}
\textit{\blindtext[1]}

\section{Practice}

\blindtext[1]


\end{document}