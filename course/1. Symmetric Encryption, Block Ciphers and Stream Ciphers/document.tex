\documentclass[12pt]{article}
%%  ==================================================
%%  Settings
%%  ==================================================

\usepackage[a4paper, inner=1cm, outer=1cm, top=2cm, bottom=2cm]{geometry}
\usepackage{fontspec}
\usepackage{blindtext}
\usepackage{enumitem}
\usepackage{fancyhdr}
\usepackage[hidelinks]{hyperref}
\usepackage{index}
\usepackage{listings}
\usepackage{color}
\usepackage{graphicx}
\usepackage{caption}
\usepackage{tikz}           % for drawing external link icon

%%  ==================================================
%%  Styles
%%  ==================================================

\setmainfont{Arimo}

\pagestyle{fancy}
\fancyfoot{}
\raggedbottom

\date{}

\renewcommand{\headrulewidth}{1pt}
\renewcommand{\footrulewidth}{0pt}

\definecolor{dkgreen}{rgb}{0,0.6,0}
\definecolor{red}{rgb}{0.9,0.1,0.1}
\definecolor{gray}{rgb}{0.5,0.5,0.5}
\definecolor{mauve}{rgb}{0.58,0,0.82}


\newcommand{\ExternalLink}{%
    \tikz[x=1.2ex, y=1.2ex, baseline=-0.05ex]{% 
        \begin{scope}[x=1ex, y=1ex]
            \clip (-0.1,-0.1) 
                --++ (-0, 1.2) 
                --++ (0.6, 0) 
                --++ (0, -0.6) 
                --++ (0.6, 0) 
                --++ (0, -1);
            \path[draw, 
                line width = 0.5, 
                rounded corners=0.5] 
                (0,0) rectangle (1,1);
        \end{scope}
        \path[draw, line width = 0.5] (0.5, 0.5) 
            -- (1, 1);
        \path[draw, line width = 0.5] (0.6, 1) 
            -- (1, 1) -- (1, 0.6);
        }
    }

\newfontfamily{\codefont}{Input Mono} % You need to have this ttf in your os

\lstset{frame=tb,
  aboveskip=3mm,
  belowskip=3mm,
  showstringspaces=false,
  columns=flexible,
  basicstyle={\codefont\small},
  numbers=none,
  numberstyle=\tiny\color{red},
  keywordstyle=\color{blue},
  commentstyle=\color{dkgreen},
  stringstyle=\color{mauve},
  breaklines=true,
  breakatwhitespace=true,
  tabsize=3
}

\newcommand{\documenttitle}{1. Symmetric Encryption, Block Ciphers and Stream Ciphers}
\fancyhead[L]{\documenttitle}
\title{\documenttitle}

\begin{document}
\maketitle

Learn the fundamentals of symmetric encryption, as well as the differences between block ciphers and stream ciphers, and their respective modes of operation. This lesson will not go too much in depth to keep these lessons simple, so I highly recommend you to explore each concept in your own time.

\section{Learn}

\subsection{Fundamentals}

Encrypting and decrypting information \textbf{using the same key} for both of these operations is what we call symmetric cryptography.

\begin{center}
    \includegraphics[width=0.5\textwidth]{resources/symmetric.png}
    \captionof{figure}[LOF entry]{Symmetric encryption process}
    \label{fig:symmetric}
\end{center}

A plethora of different algorithms have been developed over the years. Most commonly used to this day are \href{https://en.wikipedia.org/wiki/Advanced_Encryption_Standard}{AES \ExternalLink} (also known as Rijndael), \href{https://en.wikipedia.org/wiki/ChaCha20}{ChaCha20 \ExternalLink}, \href{https://en.wikipedia.org/wiki/ARIA_(cipher)}{ARIA \ExternalLink}, \href{https://en.wikipedia.org/wiki/Camellia_(cipher)}{Camellia \ExternalLink}, and a couple of others. You may have also heard about ciphers like \href{https://en.wikipedia.org/wiki/RC4}{RC4 \ExternalLink}, \href{https://en.wikipedia.org/wiki/Data_Encryption_Standard}{DES \ExternalLink} and \href{https://en.wikipedia.org/wiki/Triple_DES}{3DES \ExternalLink}, however they are considered to be insecure and therefore were deprecated. 

\subsection{Where is it used?}

Whenever we talk about encryption protocols, you can safely assume symmetric encryption is used. For example, TLS secures web access by combining symmetric and asymmetric ciphers, key exchange, authentication, and hashing (in what we call a ciphersuite). Symmetric ciphers are also the standard for encrypting large data volumes. VeraCrypt, a disk encryption utility, uses AES, Camellia, Kuznyechik, where as BitLocker uses AES. SSH used RC4, DES, and 3DES in the past, but now primarily relies on AES and ChaCha20.

\subsection{Symmetric vs. Asymmetric}

Distinctive differences between symmetric and asymmetric cryptography have been described in the table below.

\begin{center}
{\renewcommand{\arraystretch}{1.7}
\begin{tabular}{| m{3cm} | m{7cm}| m{7cm} |}
    \hline
    & Symmetric & Asymmetric \\
    \hline
    Key aspects & Symmetric cryptography uses \textbf{one key} to both encrypt and decrypt information. & Asymmetric cryptography uses \textbf{two mathematically related keys}: a public key (shared with others) and a private key (kept secret). The public key encrypts messages, while the private key decrypts them. \\
    \hline
    Key security & Anyone with access to the key can decrypt the ciphertext. & Only the holder of the private key can decrypt information. Access to the public key does not enable an adversary to decrypt messages. \\
    \hline
    Use cases & Encrypting large volumes of data (e.g., files, drives, streams). & Encrypting smaller volumes of data (e.g., messages, emails, keys), creating digital signatures, and key exchange. \\
    \hline
    Operation speed & Encryption and decryption are usually fast and efficient, with lower computational requirements. & Asymmetric operations are slower due to larger key sizes, computational overhead, and mathematical complexity. \\
    \hline
    Assurances & Symmetric ciphers provide only \textbf{confidentiality}. & Asymmetric ciphers provide \textbf{confidentiality}, \textbf{integrity}, and \textbf{non-repudiation}. \\
    \hline

\end{tabular}
}
\end{center}

You should understand that symmetric cryptography—despite its disadvantages—plays an important role in modern cryptography use cases as well as in hybrid cryptography scenarios.

\subsection{Types of ciphers}

\subsubsection{Block ciphers}
\textit{\blindtext[1]}

\subsubsection{Stream ciphers}
\textit{\blindtext[1]}

\section{Practice}

\blindtext[1]


\end{document}